%%%%%%%%%%%%%%%%%%%%%%%%%%%%%%%%%%%%%%%%%%%%%%%%%%%%%%%%%%%%%%%%%%%%%%%%%
%%
%W  intro.tex          Radiroot documentation             Andreas Distler
%%
%H  $Id:$
%%
%Y  2005
%%

%%%%%%%%%%%%%%%%%%%%%%%%%%%%%%%%%%%%%%%%%%%%%%%%%%%%%%%%%%%%%%%%%%%%%%%%%
\Chapter{Introduction}

The main functionality of this package is to solve a rational polynomial by
radicals and display the solution. That is possible iff the Galois group of
the polynomial -- a permutation group on its roots -- is solvable. This fact
has first been discovered by \'Evariste Galois (1811 -- 1832), on whose ideas
this implementation is based. The implemented algorithm is discribed in
\cite{Distler05}.

The package creates a LaTex file for the radical expression. Therefore you
need a Latex compiler and the dvi viewer xdvi, to use the main
functionality.  

In addition to the readout you get several results in {\GAP}. Some of them can
be computed on their own. This are the splitting field of a rational polynomial
and its Galois group as a permutation on the roots.

This package uses the interface to KANT \cite{KANT} in the package Alnuth to
factorize polynomials over algebraic numberfields. This functionality must
be available to use the functions in {\Radiroot}.  


%%%%%%%%%%%%%%%%%%%%%%%%%%%%%%%%%%%%%%%%%%%%%%%%%%%%%%%%%%%%%%%%%%%%%%%%%
%%
%E
